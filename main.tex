\documentclass[10pt]{article}
\usepackage[mode=buildnew]{standalone}
% Layout Options
\usepackage[margin = 1.0in]{geometry}
\geometry{letterpaper}

\usepackage{acjpreamble}
\usepackage{listings}

 \usepackage[normalem]{ulem}

\lstset{
tabsize = 4, %% set tab space width
    showstringspaces = false, %% prevent space marking in strings, string is defined as the text that is generally printed directly to the console
    numbers = left, %% display line numbers on the left
    commentstyle = \color{gray}, %% set comment color
    keywordstyle = \color{blue}, %% set keyword color
    stringstyle = \color{red}, %% set string color
    rulecolor = \color{black}, %% set frame color to avoid being affected by text color
    basicstyle = \small \ttfamily , %% set listing font and size
    breaklines = true, %% enable line breaking
    numberstyle = \tiny,
}


% \setlength{\headheight}{15.2pt}

\begin{document}
\thispagestyle{empty}
\begin{titlepage}
    \centering
    {\Huge Project 2 \par \vspace{0.5cm}}
    %{\huge Operating Systems \par \vspace{0.5cm}}
    {\Large A CPU Scheduling Simulator \par}
    {\normalsize   \par}

    \vfill
    {\normalsize Operating Systems \par}
    {\normalsize Department of Computer Science \par}
    {\normalsize Colorado School of Mines \par}
    {\normalsize \today \par}
    \thispagestyle{empty}
    \thispagestyle{empty}
\end{titlepage}

\clearpage
\pagestyle{fancy}
\lhead[Project 2 --- CPU Scheduling Simulator]{Project 2 --- CPU Scheduling Simulator}
\rhead[CSCI 442 --- Operating Systems]{CSCI 442 --- Operating Systems}

\newcommand{\assigndate}{March 7, 2020}
\newcommand{\donedate}{23:59 \sout{March 16, 2020} March 30, 2020}
\newcommand{\dtwodate}{23:59 \sout{April 6, 2020} April 13, 2020}
\newcommand{\dthreedate}{23:59 March 30, 2020}
%--------------------------------------------------
%	INTRODUCTION
%--------------------------------------------------


\begin{center}
{\large Assigned Date: \assigndate} \\ \vspace{0.35em}
{\Large Deliverable 1 Due: \donedate} \\ \vspace{0.325em}
{\Large Deliverable 2 Due: \dtwodate}
\end{center}

\section*{Introduction}
The goal of this project is to develop a CPU scheduling simulation that will complete the execution of a
group of multi-threaded processes. It will support several different scheduling algorithms. The
user can then specify which one to use via a command-line flag. At the end of execution, your program will calculate and
display several performance criteria obtained by the simulation.

Learning goals:
\begin{enumerate}
    \item You will have better familiarity with one of the main roles of any operating system: process scheduling.
    \item You will become familiar with event-driven simulation.
    \item You will understand the performance implications of using different scheduling algorithms.
    In the future, you can reuse these concepts (scheduling, simulation, etc) in any optimization task you are given in your professional life.
\end{enumerate}

This project must be implemented in C++, and it must execute correctly on the computers in the Alamode lab.

\subfile{files/simulation-info.tex}
\subfile{files/scheduling-algorithms.tex}
\subfile{files/next-event-simulation.tex}
\subfile{files/deliverables.tex}
\subfile{files/grading.tex}
\subfile{files/getting-started.tex}
%\subfile{files/requirements.tex}
\subfile{files/input-format.tex}
\subfile{files/command-line.tex}
\subfile{files/output-format.tex}

% \subfile{files/starter-code.tex}


\clearpage
\appendix
\appendixpage

\section{Deliverable 1 Simulation}
\label{app:d-one-sim}

\lstinputlisting{inputs/d1_simulation}

\section{Example Simulation Output}
\label{app:example-simple-output}

For the following simulation:
\lstinputlisting{inputs/simple_simulation}

this was output:

\lstinputlisting{outputs/simple_sim_output}

\subfile{files/func-apps.tex}

\end{document}

%--------------------------------------------------
%	References
%--------------------------------------------------

% If you have references they go here, just comment out both lines

%\bibliography{references}
%\bibliographystyle{IEEEtran}

% \clearpage
% \appendix
% \appendixpage

% \end{document}
