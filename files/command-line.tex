\documentclass[main.tex]{subfiles}
\begin{document}
\section{Command Line Parsing}
\label{sec:cmd-line}

Your simulation must support invocation in the format specified below, including the following command line flags:

\begin{verbatim}
./cpu-sim [flags] [simulation_file]

-h, --help
    Print a help message on how to use the program.
    
-m, --metrics
    If set, output general metrics for the simulation.
    
-s, --time_slice [positive integer]
    The time slice for preemptive algorithms.
    
-t, --per_thread
    If set, outputs per-thread metrics at the end of the simulation.
    
-v, --verbose
    If set, outputs all state transitions and scheduling choices.
    
-a, --algorithm <algorithm>
    The scheduling algorithm to use. Valid values are:
        FCFS: first come, first served (default)
        RR: round robin scheduling
        PRIORITY: priority-based scheduling
        MLFQ: multi-level feedback queue
        CUSTOM: a custom algorithm
\end{verbatim}

Users should be able to pass any flags together, in any order, provided that:

\begin{itemize}
    \item If the \texttt{--help} flag is set, a help message is printed to \texttt{stdout} and the program \textit{immediately} exits.
    \item If \texttt{--time\_slice} is set, it must be followed immediately by a positive integer.
    \item If \texttt{--algorithm} is set, it must be followed immediately by an algorithm choice.
    \item If \texttt{--algorithm} is not set, your program shall default to using first come, first served as its scheduling algorithm.
    \item If a filename is not provided, the program shall read in from \texttt{stdin}.
\end{itemize}

Any improper command line input should cause your program to print the help message and then immediately exit. Information on proper output formatting can be found in Section \ref{sec:out-format}.

You are \emph{strongly encouraged} to use the \texttt{getopt} family of functions to perform the command line parsing. Information on \texttt{getopt} can be found here: \url{http://man7.org/linux/man-pages/man3/getopt.3.html}

\end{document}