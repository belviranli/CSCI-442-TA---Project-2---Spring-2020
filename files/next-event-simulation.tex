\documentclass[main.tex]{subfiles}
\begin{document}
\section{Next-Event Simulation}
\label{sec:ne-sim}
Your simulation structure must follow the next-event pattern. At any given time, the simulation is in a single
state. The simulation state can only change at event times, where an event is defined as an occurrence
that may change the state of the system.

Since the simulation state only changes at an event, the ”clock” can be advanced to the next scheduled
event–regardless of whether the next event is 1 or 1,000,000 time units in the future. This is why it is called
a ”next-event” simulation model. In our case, time is measured in simple ”units”.
Your simulation must support the following event types:

\begin{itemize}
    \item \texttt{THREAD ARRIVED}: A thread has been created in the system.
    \item \texttt{THREAD DISPATCH COMPLETED}: A thread switch has completed, allowing a new thread to start executing on the CPU.
    \item \texttt{PROCESS DISPATCH COMPLETED}: A process switch has completed, allowing a new thread to start executing on the CPU.
    \item \texttt{CPU BURST COMPLETED}: A thread has finished one of its CPU bursts and has initiated an I/O request.
    \item \texttt{IO BURST COMPLETED}: A thread has finished one of its I/O bursts and is once again ready to be executed.
    \item \texttt{THREAD COMPLETED}: A thread has finished the last of its CPU bursts.
    \item \texttt{THREAD PREEMPTED}: A thread has been preempted during execution of one of its CPU bursts.
    \item \texttt{DISPATCHER INVOKED}: The OS dispatcher routine has been invoked to determine the next thread to be run on the CPU.
\end{itemize}

The main loop of the simulation should consist of processing the next event, perhaps adding more future
events in the queue as a result, advancing the clock (by taking the next scheduled event from the front of
the event queue), and so on until all threads have terminated. See Figure \ref{fig:des} for an illustration of the event simulation. Rounded rectangles indicate functions that you will need to implement to handle the associated event types.

\begin{figure}
    \centering
    \includestandalone[width=0.90\textwidth]{files/diagrams/des-diagram.tex}
    \caption{A high level illustration of the next-event simulation. In the starter code, all of this functionality is to be implemented within the \texttt{Simulation} class. Rounded rectangles represent functions, while diamonds are decisions that lead to different actions being taken. For example, if the event type is determined to be \texttt{THREAD ARRIVED}, then the \texttt{handle\_thread\_arrived(event)} function should be called.}
    \label{fig:des}
\end{figure}


\subsection{Event Queue}

Events are scheduled via an event queue. The event queue is a priority queue that contains future events;
the priority of each item in the queue corresponds to its scheduled time, where the event with the highest
"priority" (at the front of the queue) is the one that will happen next.

To determine the next event to handle, a priority queue is used to sort the events. For this project, the event queue should sort based on these criteria:
\begin{itemize}
    \item The time the event occurs. The earliest time comes first (time 3 comes before time 12).
    \item If two events have the time, then the tie breaker should be the events' number: as each new event is created, it should be assigned a number representing how many events have been created. For example, the first event in the simulation should be given the number \texttt{0}, the second the number \texttt{1}, and so on. The earliest number should come first (event number 6 comes before event number 7).
\end{itemize}



\end{document}