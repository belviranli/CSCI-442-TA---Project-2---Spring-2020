\documentclass[main.tex]{subfiles}
\begin{document}

\section{Requirements and Reference}
\label{sec:req_refs}

\begin{itemize}
    \item To compile your code, the grader should be to \texttt{cd} into it and simply type \texttt{make}.
    \item Your simulation should be able to be executed by typing \texttt{/.cpu-sim}.
    \item Any improper command line input should cause your program to print the help message and then immediately exit.
    \item Do not modify the directory structure provided in the starter code.
    \item The \texttt{author} file must be present in all submissions, as detailed in Section \ref{sec:deliverables}.
    \item \texttt{submission-details/} contains everything that is required, as detailed in Section \ref{sec:deliverables}.
    \item Your project must execute correctly on the Alamode computers. That is where it will be graded.
    \item Use good formatting skills. A well formatted project will not only be easier to work in and debug, but it will also make for a happier grader.
    \item Your final submission must contain a \texttt{README} file with the following information:
    \begin{itemize}
        \item Your name
        \item A list of all the files in your submission and what each does
        \item Any unusual or interesting features in your program
        \item Approximate number of hours you spent on the project
    \end{itemize}
    \item If you have implemented the extra credit algorithm and would like to receive credit for them, make sure that you have followed the instructions for creating the files as detailed in Section \ref{sec:deliverables}.
    \item This list is subject to change as we proceed through the project.
\end{itemize}

\end{document}